\documentclass[11pt, english, fleqn, DIV=15, headinclude]{scrartcl}

\usepackage[bibatend, color]{header}

\hypersetup{
    pdftitle=
}

% http://tex.stackexchange.com/a/42620
\usepackage{pifont}% http://ctan.org/pkg/pifont
\newcommand{\yes}{\ding{51}}%
\newcommand{\no}{\ding{55}}%

%\subject{}
\title{Summary of notations}
%\subtitle{}
\author{
    Martin Ueding \\ \small{\href{mailto:mu@martin-ueding.de}{mu@martin-ueding.de}}
}

\begin{document}

\maketitle

\begin{tabular}{cccccc}
    Class & Examples & italic & serif & bold & underline \\
    \midrule
    Scalar & $x$, $\lambda$                & \yes & \yes & \no  & \no  \\
    Vector & $\vec v$, $\vec F$            & \yes & \yes & \yes & \no  \\
    Matrix & $\vec A$, $\vec M$            & \yes & \yes & \yes & \no  \\
    4-Vector & $\four v$, $\four P$        & \yes & \yes & \yes & \yes \\
    Tensor & $\tens G$, $\tens T$          & \yes & \no  & \yes & \no  \\
    Field & $\R$, $\C$                     & \no  & \no  & \yes & \no  \\
\end{tabular}

\paragraph{Why the underline?}

I try to stick close to ISO 80000-2. One problem I have in Physics are the
three-vectors and four-vectors. Mathematically they are just from $\R^3$ and
$\R^4$ respectively and they are both vectors. One sometimes has equations like
$\four x \cdot \four p = x^0 p^0 - \vec x \cdot \vec p$. This would not be
clear if one write $\vec x \cdot \vec p$ or $x \cdot p$ on the left side, I
think.


\end{document}

% vim: spell spelllang=en tw=79
